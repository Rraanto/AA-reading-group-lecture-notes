% --- LaTeX Lecture Notes Template - S. Venkatraman ---

% --- Set document class and font size ---

\documentclass[a4paper, 12pt]{article}

% for french or other lang
% \usepackage[francais]{babel}

% --- Package imports ---

% Extended set of colors
\usepackage[dvipsnames]{xcolor}

% for boxes around stuff
\usepackage{mdframed}

\usepackage{
  amsmath, bm, amsthm, amssymb, mathtools, dsfont, units,          % Math typesetting
  graphicx, wrapfig, subfig, float,                            % Figures and graphics formatting
  listings, color, inconsolata, pythonhighlight,               % Code formatting
  fancyhdr, sectsty, hyperref, enumerate, enumitem, framed }   % Headers/footers, section fonts, links, lists

% lipsum is just for generating placeholder text and can be removed
\usepackage{hyperref, lipsum} 

% --- Fonts ---

% \usepackage{newpxtext, newpxmath, inconsolata}

% --- Page layout settings ---

% Set page margins
\usepackage[left=1.9cm, right=1.32cm, top=1.9cm, bottom=3.67cm, headsep=0.95cm, footskip=0.95cm]{geometry}

% Anchor footnotes to the bottom of the page
\usepackage[bottom]{footmisc}

% Set line spacing
\renewcommand{\baselinestretch}{1}

% Set spacing between paragraphs
\setlength{\parskip}{1.3mm}

% Allow multi-line equations to break onto the next page
\allowdisplaybreaks

% --- Page formatting settings ---

% Set image captions to be italicized
\usepackage[font={it,footnotesize}]{caption}

% Set link colors for labeled items (blue), citations (red), URLs (orange)
\hypersetup{colorlinks=true, linkcolor=RoyalBlue, citecolor=RedOrange, urlcolor=ForestGreen}

% Set font size for section titles (\large) and subtitles (\normalsize) 
\usepackage{titlesec}
\titleformat{\section}{\large\bfseries}{{\fontsize{19}{19}\selectfont\textreferencemark}\;\; }{0em}{}
\titleformat{\subsection}{\normalsize\bfseries\selectfont}{\thesubsection\;\;\;}{0em}{}

% Enumerated/bulleted lists: make numbers/bullets flush left
%\setlist[enumerate]{wide=2pt, leftmargin=16pt, labelwidth=0pt}
\setlist[itemize]{wide=0pt, leftmargin=16pt, labelwidth=10pt, align=left}

% --- Table of contents settings ---

\usepackage[subfigure]{tocloft}

% Reduce spacing between sections in table of contents
\setlength{\cftbeforesecskip}{.9ex}

% Remove indentation for sections
\cftsetindents{section}{0em}{0em}

% Set font size (\large) for table of contents title
\renewcommand{\cfttoctitlefont}{\large\bfseries}

% Remove numbers/bullets from section titles in table of contents
\makeatletter
\renewcommand{\cftsecpresnum}{\begin{lrbox}{\@tempboxa}}
\renewcommand{\cftsecaftersnum}{\end{lrbox}}
\makeatother

% --- Set path for images ---

\graphicspath{{Images/}{../Images/}}

% --- Math/Statistics commands ---

% Add a reference number to a single line of a multi-line equation
% Usage: "\numberthis\label{labelNameHere}" in an align or gather environment
\newcommand\numberthis{\addtocounter{equation}{1}\tag{\theequation}}

% Shortcut for bold text in math mode, e.g. $\b{X}$
\let\b\mathbf

% Shortcut for bold Greek letters, e.g. $\bg{\beta}$
\let\bg\boldsymbol

% Shortcut for calligraphic script, e.g. %\mc{M}$
\let\mc\mathcal

% \mathscr{(letter here)} is sometimes used to denote vector spaces
\usepackage[mathscr]{euscript}

% for doble brackets 
\usepackage{stmaryrd}

% Convergence: right arrow with optional text on top
% E.g. $\converge[p]$ for converges in probability
\newcommand{\converge}[1][]{\xrightarrow{#1}}

% Weak convergence: harpoon symbol with optional text on top
% E.g. $\wconverge[n\to\infty]$
\newcommand{\wconverge}[1][]{\stackrel{#1}{\rightharpoonup}}

% Equality: equals sign with optional text on top
% E.g. $X \equals[d] Y$ for equality in distribution
\newcommand{\equals}[1][]{\stackrel{\smash{#1}}{=}}

% Normal distribution: arguments are the mean and variance
% E.g. $\normal{\mu}{\sigma}$
\newcommand{\normal}[2]{\mathcal{N}\left(#1,#2\right)}

% Uniform distribution: arguments are the left and right endpoints
% E.g. $\unif{0}{1}$
\newcommand{\unif}[2]{\text{Uniform}(#1,#2)}

% Independent and identically distributed random variables
% E.g. $ X_1,...,X_n \iid \normal{0}{1}$
\newcommand{\iid}{\stackrel{\smash{\text{iid}}}{\sim}}

% Sequences (this shortcut is mostly to reduce finger strain for small hands)
% E.g. to write $\{A_n\}_{n\geq 1}$, do $\bk{A_n}{n\geq 1}$
\newcommand{\bk}[2]{\{#1\}_{#2}}

% Math mode symbols for common sets and spaces. Example usage: $\R$
\newcommand{\R}{\mathbb{R}}	% Real numbers
\newcommand{\C}{\mathbb{C}}	% Complex numbers
\newcommand{\Q}{\mathbb{Q}}	% Rational numbers
\newcommand{\Z}{\mathbb{Z}}	% Integers
\newcommand{\N}{\mathbb{N}}	% Natural numbers
\newcommand{\F}{\mathcal{F}}	% Calligraphic F for a sigma algebra
\newcommand{\El}{\mathcal{L}}	% Calligraphic L, e.g. for L^p spaces

% Math mode symbols for probability
\newcommand{\pr}{\mathbb{P}}	% Probability measure
\newcommand{\E}{\mathbb{E}}	% Expectation, e.g. $\E(X)$
\newcommand{\var}{\text{Var}}	% Variance, e.g. $\var(X)$
\newcommand{\cov}{\text{Cov}}	% Covariance, e.g. $\cov(X,Y)$
\newcommand{\corr}{\text{Corr}}	% Correlation, e.g. $\corr(X,Y)$
\newcommand{\B}{\mathcal{B}}	% Borel sigma-algebra

% Other miscellaneous symbols
\newcommand{\tth}{\text{th}}	% Non-italicized 'th', e.g. $n^\tth$
\newcommand{\Oh}{\mathcal{O}}	% Big-O notation, e.g. $\O(n)$
\newcommand{\1}{\mathds{1}}	% Indicator function, e.g. $\1_A$

% Additional commands for math mode
\DeclareMathOperator*{\argmax}{argmax}		% Argmax, e.g. $\argmax_{x\in[0,1]} f(x)$
\DeclareMathOperator*{\argmin}{argmin}		% Argmin, e.g. $\argmin_{x\in[0,1]} f(x)$
\DeclareMathOperator*{\spann}{Span}		% Span, e.g. $\spann\{X_1,...,X_n\}$
\DeclareMathOperator*{\bias}{Bias}		% Bias, e.g. $\bias(\hat\theta)$
\DeclareMathOperator*{\ran}{ran}			% Range of an operator, e.g. $\ran(T) 
\DeclareMathOperator*{\dv}{d\!}			% Non-italicized 'with respect to', e.g. $\int f(x) \dv x$
\DeclareMathOperator*{\diag}{diag}		% Diagonal of a matrix, e.g. $\diag(M)$
\DeclareMathOperator*{\trace}{trace}		% Trace of a matrix, e.g. $\trace(M)$
\DeclareMathOperator*{\supp}{supp}		% Support of a function, e.g., $\supp(f)$

% Numbered theorem, lemma, etc. settings - e.g., a definition, lemma, and theorem appearing in that 
% order in Lecture 2 will be numbered Definition 2.1, Lemma 2.2, Theorem 2.3. 
% Example usage: \begin{theorem}[Name of theorem] Theorem statement \end{theorem}
\theoremstyle{definition}
\newtheorem{theorem}{Theorem}[section]
\newtheorem{proposition}[theorem]{Proposition}
\newtheorem{lemma}[theorem]{Lemma}
\newtheorem{corollary}[theorem]{Corollary}
\newtheorem{definition}[theorem]{Definition}
\newtheorem{example}[theorem]{Example}
\newtheorem{remark}[theorem]{Remark}
\newmdtheoremenv{boxedTheorem}{Theorem}
\newmdtheoremenv{boxedProposition}{Proposition}
\newmdtheoremenv{boxedLemma}{Lemma}
\newmdtheoremenv{boxedCorollary}{Corollary}
\newmdtheoremenv[rightline=false, topline=false, leftline=true, bottomline=false]{boxedDefinition}{Definition}
\newmdtheoremenv{boxedExample}{Example}
\newmdtheoremenv{boxedRemark}{Remark}
\newmdtheoremenv{boxedProblem}{Problem}

% Un-numbered theorem, lemma, etc. settings
% Example usage: \begin{lemma*}[Name of lemma] Lemma statement \end{lemma*}
\newtheorem*{theorem*}{Theorem}
\newtheorem*{proposition*}{Proposition}
\newtheorem*{lemma*}{Lemma}
\newtheorem*{corollary*}{Corollary}
\newtheorem*{definition*}{Definition}
\newtheorem*{example*}{Example}
\newtheorem*{remark*}{Remark}
\newtheorem*{claim}{Claim}

% --- Left/right header text (to appear on every page) ---

% Do not include a line under header or above footer
\pagestyle{fancy}
\renewcommand{\footrulewidth}{0pt}
\renewcommand{\headrulewidth}{0pt}

% Right header text: Lecture number and title
\renewcommand{\sectionmark}[1]{\markright{#1} }
\fancyhead[R]{\small\textit{\nouppercase{\rightmark}}}

% Left header text: Short course title, hyperlinked to table of contents
\fancyhead[L]{\hyperref[sec:contents]{\small Reading group lecture notes}}

% For roman numerals 
\makeatletter
\newcommand{\rmnum}[1]{\romannumeral #1}
\newcommand{\Rmnum}[1]{\expandafter\@slowromancap\romannumeral #1@}
\makeatother

% --- Document starts here ---

\begin{document}
\bibliographystyle{plain}

% --- Main title and subtitle ---

\title{Reading group lecture notes \\[1em]
\normalsize on Abstract Algebra}

% --- Author and date of last update ---

\author{\normalsize}
\date{\normalsize\vspace{-1ex} Last updated: \today}

% --- Add title and table of contents ---

\maketitle
\tableofcontents\label{sec:contents}

% --- Main content: import lectures as subfiles ---

% TeX root = ../Main.tex

\section*{Remarks}

These are lecture notes taken for \href{https://discord.gg/5bVSwQQR}{this Abstract Algebra reading group}\footnote{https://discord.gg/5bVSwQQR}, based on Michael Artin's Algebra \cite{artin2011algebra}, and following \href{https://wayback.archive-it.org/3671/20150528171650/https://www.extension.harvard.edu/open-learning-initiative/abstract-algebra}{these free online lecture videos}\footnote{https://wayback.archive-it.org/3671/20150528171650/https://www.extension.harvard.edu/open-learning-initiative/abstract-algebra}

The notes, problems and solutions are added to the document as the reading group progresses through the course.

\newpage

% TeX root: ../Main.tex

\section{Lecture 1}

\subsection{Groups and subgroups}
\begin{boxedDefinition}[Group]\label{def:group-def}
    
Let $G$ be a set together with a composition law, denoted $\cdot$ , following the following properties:
\begin{enumerate}
\item\label{def:group-def-associativity} For any $a, b, c$ in $G$, $(a \cdot b) \cdot c = a \cdot (b \cdot c)$ \textit{(Associativity of the law)}
\item\label{def:group-def-neutral-element} There exists an identity element $1_G$ in $G$ such that for any $a$ in $G$, $a\cdot 1_G = 1_G\cdot a = a$ \textit{(Identity element)}
\item \label{def:group-def-inverse} Each element $a$ in $G$ has an inverse $b$ satisfying: $a \cdot b = b \cdot a = 1_G$ \textit{(Inverse element)}
\end{enumerate} 
\end{boxedDefinition}

\begin{boxedDefinition}[Order]\label{def:group-order}
    The \textit{\textbf{order}} of a group $(G, \cdot)$, denoted $|G|$, is the number of elements that it contains.
\end{boxedDefinition}
 If $|G|$ is finite, $G$ is said to be a \textit{\textbf{Finite group}}, if not then $G$ is an \textit{\textbf{Infinite group}}


    \begin{example}
        
        \item []\label{ex:group-examples}
        \begin{itemize}
        \item[$(\Z, +)$]: The set of integers, with addition as its law of composition, -- the \textit{additive group of integers}. 
        \item[$(\R, +)$]: The set of real numbers, with addition as its law of composition -- the \textit{additive group of real numbers}
        \item[$(\R^*, \cdot)$]: The set of \textbf{nonzero} real numbers with multiplication as its law of composition -- the multiplicative group
        \item[$(\C, +), (\C^*, \cdot)$]: Analogous groups, where $\C$ (resp. $\C^*$), the set of complex numbers (resp. nonzero complex numbers), replaces $\R$ (resp. $\R^\times$).
        \item[$(GL_n, \cdot)$]: The set of \textit{\textbf{Invertible}}  $n \times n$ matrices, with the matrix multiplication as its composition law, also named the \textit{general linear group}. 
        \item[$(Sym(X), \circ)$]: The set of all \textbf{\textit{bijective}} maps $f: X \longrightarrow X$, together with the \textit{composition of function} as a law, forms a group, usually called the \textit{symmetric group of the set $X$}. 
        \end{itemize}

\end{example}

\begin{remark}

    \begin{enumerate}
        \item[]
        \item In practice, to refer to a group $(G, \cdot)$, where $\cdot$ is its composition law, we usually just use "$G$", especially when the set has a "natural" law (e.g. addition for integers, composition of function for sets of functions). When an arbitrary group is discussed, it is also common to refer to the law as "multiplication", this does not mean that the involved set is a set of numbers.
        \item The identity element is usually denoted just $1$, $1_G$, $e$, or $e_G$.
        \item The inverse of an element $a$ in $G$ is denoted $a^{-1}$. 
        \item If $G$ is a group whose law satisfies $ab = ba$ for any $a, b$ in $G$, then $G$ is said to be \textit{\textbf{Abelian}}. The list of examples above contains only one non-Abelian groups. 
        
        \item We notice that the constraint to take \textit{only} invertible (resp. nonzero) matrices (resp. real numbers) is necessary in order to fulfill property \ref{def:group-def-inverse} of a group.
More generally, in possession of a set $S$, with an associative law, and an element who does not affect the law\footnote{element $e$ satisfying $ae = ea = a$ for all $a$ in the set}, it is always possible to form a group, by considering only the subset of $S$ whose elements all have inverses in $S$ (in the sense of \ref{def:group-def}.\ref{def:group-def-inverse}).
       
    \end{enumerate}
\end{remark} 

\begin{boxedProposition} [Cancellation law]
    let $G$ be a group and $a, b, c$ be elements of $G$.\begin{itemize}
     \item If $ab = ac$ or $ba = ca$, then $b=c$
     \item If $ab = a$ or $ba = a$, then $b=1$
    \end{itemize}
\end{boxedProposition}
\begin{proof}
    Suppose $ab = ac$, then: $\underbrace{(a^{-1})ab}_{=1b} = \underbrace{(a^{-1})ac}_{=1c}$.
\end{proof}
The other proofs are analogous

\begin{boxedDefinition}[Subgroups]\label{def:subgroup}
    let $G$ be a group. A subset $H$ of $G$ is called a \textit{subgroup} of $G$ if it satisfies the following conditions:
    \begin{enumerate}  
        \item \label{def:subgroup-closure} for any $h$ and $k$ in $H$, $hk$ is in $H$. \textit{($H$ is "closed under the law")}
        \item \label{def:subgroup-identity} the identity element of $G$, $1_G$ is contained in $H$.
        \item \label{def:subgroup-inverse} for any $h$ in $H$, its inverse $h^{-1}$ is also contained in $H$
    \end{enumerate}
\end{boxedDefinition}

\begin{boxedRemark} \label{rem:induction-closure}
    If $H$ is a subgroup of $G$, then the closure property implies that if an element $a$ is in $H$, then for any positive integer $n$, $a^n$ = $a \cdot \hdots \cdot a$ ($n$ times) is also in $G$ by using induction: 
\end{boxedRemark}
\begin{proof}
    Let $a$ be an element in a subgroup $H$ of $G$. 

    ($n = 1$): $a^1 = a$ and $a$ is in $H$. 

    Let $k > 1$ and assume $a^k$ is in $H$. By this induction assumption, $a^k$ is in $H$, and since $a$ is in $H$, by closure : $a^k \cdot a = \underbrace{a \cdot \hdots \cdot a}_{k times} = a^{k+1}$ is also in $H$.
\end{proof}

\begin{example}
    \begin{itemize} \label{ex:subgroups-ex}
        \item[]
 \item For any group $G$, $G$ is a subgroup itself, and so is the subset $\{1_G\}$, the latter is called \textit{the trivial subgroup}
        \item The set of \textit{even integers} is a subgroup of the additive integer group: $\Z2 = \{k2 | k \in \Z\} \subset \Z$.
        Note that here the law is addition, the identity element of the group is $0$, and the inverse of an element $p$ is $(-p)$. Keeping these in mind, the properties to check are if $0$ belongs to the subset, if the inverse $-a$ of an even integer is even, and if the sum $a + b$ of two even integers $a$ and $b$ is even
        
        (Question: \textit{Is the set of \textbf{odd} integers a \textbf{subgroup} of $\Z$?} \label{ex:subgroup-even-int})
    \item The set of invertible matrices \textit{with determinant 1}: $\{M \in GL_n | det(M) = 1\} \subset GL_n$ is a subgroup of $GL_n$: it is easy to show that this subset satisfies the properties, by keeping in in mind that $det(AB) = det(A)det(B)$ for any two matrices in $GL_n$. This subgroup is also denoted $SL_n$ and is called the \textit{Special linear group}
    \item The set of complex numbers, whose modulus is equal to 1, is a subgroup of the multiplicative group of complex numbers: $\{z \in \C | |z| = 1\} \subset \C^*$. Also called \textit{Circle group}, since its elements correspond to the points of the complex plane who lie on the unit circle.
    \end{itemize}
\end{example}
The two last examples are particular cases of a more general way to \textit{find} subgroups of a given group, by using a mapping from a group to another one and adding some additional constraints(here they are $det: GL_n \longrightarrow \R^*$, and $|\cdot|: \C^* \longrightarrow \R^*$). (section \ref{sec:morphisms} contains the details).

\subsection{$(\Z, +)$ and its subgroups}
We will keep in mind what has been explained on the additive integer group in the second example from second example in \ref{ex:subgroups-ex}.

The following theorem gives a characterization\footnote{a precise criteria that is particular to} of the subgroups of $\Z$
\begin{boxedTheorem}[Subgroups of $(\Z, +)$]
    Let $S$ be a subgroup of $(\Z, +)$ that is not trivial ($\neq \{0\}$). Then $S$ has the form $\Z a$, where $a$ is the smallest positive integer in $S$.
\end{boxedTheorem}
\begin{proof}
    Let $S$ be a non-trivial subgroup of $(\Z, +)$. By this assumption, $S$ contains an integer $n$ different from $0$, and either $n$ or $-n$ (its inverse) is positive. Since $S$ is a subgroup, both $n$ and $-n$ are in $S$, meaning $S$ necessarily contains a positive integer. Let $a$ be the smallest positive integer in $S$.

    We first show that $\Z a$ is contained in $S$: Let $k$ be an integer, $ka$ is equal to $\underbrace{a + \hdots + a}_{k terms}$ if $k > 0$, or its inverse: $-(\underbrace{a + \hdots + a}_{|k| terms})$ if $k < 0$. In either case, the sum $(a + \hdots + a)$ is in $S$\footnote{This is the additive version of remark \ref{rem:induction-closure}}, and so is its inverse. 

    Next we show that $S$ is contained in $\Z a$, in other ways we are going to show that all elements of $S$ is necessarily of the form $ka$ for some integer $k$. Let $n$ be an integer in $S$, dividing $n$ by $a$ gives us two integers $q$ and $r$ such that $n = q\cdot a + r$, where $0 \leq r < a$. Since $r = n - q\cdot a$, $r$ is in $S$, and so $r$ cannot be positive as $a$  is the smallest such integer in $S$ (by choice). Thus $r = 0$ and $n = q\cdot a \in \Z a$
\end{proof}

\subsection{Cyclic groups}
\subsection{Homomorphisms, Isomorphisms}\label{sec:morphisms}

\newpage
\section{Problems and solutions}
% Tex source: ../Problems/week_1.tex
\subsection{Week 1 problems}

\begin{boxedProblem}[2.1.7]
let $S$ be any set. Prove that the law of composition defined by $ab = a$ is associative.
\end{boxedProblem}

\begin{boxedProblem}[2.2.15]
    \begin{enumerate}
        \item[]
        \item
    In the definition of subgroup, the identity element in $H$ is required to be the identity of $G$. One might require only that $H$ have an identity element, not that it is the same as the identity in $G$. Show that if $H$ has an identity at all, then it is the identity of $G$, so this definition would be equivalent to the one given.
    \item Show the analogous thing for inverses
    \end{enumerate}

\end{boxedProblem}


\begin{boxedProblem}[2.1.5]
    Assume that the equation $xyz = 1$ holds in a group $G$. Does it follow that $yzx = 1$? That $yxz = 1$?
\end{boxedProblem}

\begin{boxedProblem}[2.2.20]
    \begin{enumerate}
        \item[]
    \item Let $a, b$ be elements of an abelian group of orders $m, n$ respectively. What can you say about the order of their product $ab$?
    \item (*) Show by example that the product of elements of finite order in a nonabelian group need not have finite order.
    \end{enumerate}
\end{boxedProblem}

\begin{boxedProblem}[2.2.1]
    Determine the elements of the cyclic group generated by the matrix $\begin{pmatrix} 1 & 1 \\ -1 & 0 \end{pmatrix}$.
\end{boxedProblem}

\begin{boxedProblem}[2.4.6]
    Let $f:\R^* \longrightarrow \C^\times$ be the map $f(x) = e^{ix}$. Prove that $f$ is a homomorphism and determine its kernel and image.
\end{boxedProblem}

\begin{boxedProblem}[2.3.11]
    Prove that the set $Aut(G)$ of automorphisms of a group $G$ forms a group, the law of composition being composition of functions.
\end{boxedProblem}

\begin{boxedProblem}[2.3.12]
    Let $G$ be a group, and let $\varphi:G \longrightarrow G$ be the map $\varphi(x) = x^{-1}$.
    \begin{enumerate}
    \item Prove that $\varphi$ is bijective.
    \item Prove that $\varphi$ is an automorphism if and only if $G$ is abelian.
\end{enumerate}
\end{boxedProblem}

\begin{boxedProblem}[2.4.11]
    let $G, H$ be cyclic groups, generated by elements $x, y$. Determine the condition on the orders $m, n$ of $x$ and $y$ so that the map sending $x^i \longmapsto y^i$ is a group homomorphism.
\end{boxedProblem}

\begin{boxedProblem}[2.4.3]
    Prove that the kernel and image of a homomorphism are subgroups.
\end{boxedProblem}


\newpage
\bibliography{bibliographie}

\end{document}

% Tex source: ../Main.tex
\subsection{Week 1 problems}

\begin{boxedProblem}[2.1.7]
let $S$ be any set. Prove that the law of composition defined by $ab = a$ is associative.
\end{boxedProblem}
\begin{proof}
    Note that law as $*$. Show that $(a*b)*c = a*(b*c)$:
    $(a*b)*c = (a)*c = a*c = a = (a)*(X) = (a)*(b*c)$. Where $X$ can be replaced by anything, by definition of the law, so we just replaced it by $(b*c)$
\end{proof}

\begin{boxedProblem}[2.2.15]
    \begin{enumerate}
        \item[]
        \item
    In the definition of subgroup, the identity element in $H$ is required to be the identity of $G$. One might require only that $H$ have an identity element, not that it is the same as the identity in $G$. Show that if $H$ has an identity at all, then it is the identity of $G$, so this definition would be equivalent to the one given.
    \item Show the analogous thing for inverses
    \end{enumerate}

\end{boxedProblem}
\begin{enumerate}
    \item
\begin{proof}[]
    Let $H$ be a subgroup of $G$. suppose it has an identity element: $1_H$. It is obvious that $1_H\cdot 1_H^{-1}$ = $1_H$, but $H \subset G$ implies that $1_H$ is in $G$, so has an inverse in $G$: $1_H\cdot 1_H^{-1} = 1_G$, thus $1_G = 1_H$.
\end{proof}
\item let $x \in H$, $a$ its inverse in $H$ and $b$ its inverse in $G$.\begin{equation*}
    ax = 1_H = 1_G = bx
\end{equation*}
The cancellation law immediately implies that $a = b$.
\end{enumerate}

\begin{boxedProblem}[2.1.5]
    Assume that the equation $xyz = 1$ holds in a group $G$. Does it follow that $yzx = 1$? That $yxz = 1$?
\end{boxedProblem}
$xyz = (xy)z = 1$ means that $xy$ is the inverse of $z$, by definition of an inverse, their product still gives $1$ when commuted.

if $xyz = 1 = yxz$, by cancelling $z$, we get $xy$ = $yx$, so the second equation holds if and only if the group is abelian. 

\begin{boxedProblem}[2.2.20]
    \begin{enumerate}
        \item[]
    \item Let $a, b$ be elements of an abelian group of orders $m, n$ respectively. What can you say about the order of their product $ab$?
    \item (*) Show by example that the product of elements of finite order in a nonabelian group need not have finite order.
    \end{enumerate}
\end{boxedProblem}

\begin{boxedProblem}[2.2.1]
    Determine the elements of the cyclic group generated by the matrix $\begin{pmatrix} 1 & 1 \\ -1 & 0 \end{pmatrix}$.
\end{boxedProblem}
Immediate application of matrix products. It may be helpful to think of Proposition \ref{prop:cyclic-order-finiteness}

\begin{boxedProblem}[2.4.6] \label{exo:exponential-morphism}
    Let $f:\R^* \longrightarrow \C^\times$ be the map $f(x) = e^{ix}$. Prove that $f$ is a homomorphism and determine its kernel and image.
\end{boxedProblem}
\begin{proof}
    Noting that here $\R^*$ refers to the multiplicative group of nonzero real numbers, the fact that $f$ is a homomorphism
\end{proof}

\begin{boxedProblem}[2.3.11]
    Prove that the set $Aut(G)$ of automorphisms of a group $G$ forms a group, the law of composition being composition of functions.
\end{boxedProblem}
\begin{proof}
$Aut(G)$ is a group because: \begin{itemize}
\item for any two automorphisms $f$ and $g$, $f\circ g$ is a morphism: $(f\ circ g)(ab) = f(g(ab)) = f(g(a)g(b)) = f(g(a)) f(g(g)) = (f\circ g)(a) (f\circ g)(b)$. Moreover the composition of bijective maps is bijective, so $f\circ g$ is an automorphism, The law is well-defined. Moreover: $f\circ g \circ h = f\circ (g\circ h) = (f\circ g)\circ h$, because for an $x$ in $G$, its image through any of the three maps equals $f(g(h(x)))$. So the law is associative.
\item Let $id$ be the identity map on $G$: $id(x) = x$ for any $x \in G$, it is an automorphism. For any $f$ in $Aut(G)$: $f\circ id: x\longmapsto f(id(x)) = f(x) = id(f(x)) = (id\circ f) (x)$. So $Aut(G)$ contains an identity element: $id$. 
\item Let $f$ be in $Aut(G)$. Any bijective map has an inverse, but we need to check that $f^{-1}$ is a group homomorphism: let $a$, $b$ be two elements of $G$. note $x = f^{-1}(a), y=f^{-1}(b)$. Then: $ab$ = $f(x)f(y)$ = $f(xy)$, so $xy$ = $f^{-1}(ab) = f^{-1}(a)f^{-1}(b)$ 
\end{itemize}
\end{proof}

\begin{boxedProblem}[2.3.12]
    Let $G$ be a group, and let $\varphi:G \longrightarrow G$ be the map $\varphi(x) = x^{-1}$.
    \begin{enumerate}
    \item Prove that $\varphi$ is bijective.
    \item Prove that $\varphi$ is an automorphism if and only if $G$ is abelian.
\end{enumerate}
\end{boxedProblem}
    $\varphi$ is bijective because the definition of a group requires each element to have an unique inverse. The second point comes from the remark that $\varphi(ab)$ = $(ab)^{-1} = b^{-1}a^{-1}$ which can be $\neq \varphi(a)\varphi(b) = a^{-1}b^{-1}$ unless $G$ is abelian.

\begin{boxedProblem}[2.4.11]\label{exo:cyclic-morphism}
    let $G, H$ be cyclic groups, generated by elements $x, y$. Determine the condition on the orders $m, n$ of $x$ and $y$ so that the map sending $x^i \longmapsto y^i$ is a group homomorphism.
\end{boxedProblem}

\begin{boxedProblem}[2.4.3]\label{exo:ker-image}
    Prove that the kernel and image of a homomorphism are subgroups.
\end{boxedProblem}
\begin{proof}
    Let $f:G \longrightarrow H$ be a group homomorphism. $ker(f)$ is a subgroup of $G$ because: \begin{itemize}
        \item $1_G \in ker(f)$
        \item for any $x, y$ in $ker(f)$: $f(xy) = f(x)f(y) = 1_H1_H = 1_H$.
        \item for any $x$ in $ker(f)$: $f(x^{-1}) = f(x)^{-1} = 1_H$.
    \end{itemize}

    Note that an element h belonging to $Im(f)$ ensures the existence of $g$ in $G$ such that $f(g) = h$. $Im(f)$  is a subgroup of $H$ because: \begin{itemize}
        \item $1_H = f(1_G)\in Im(f)$
        \item for any $x=f(a), y=f(b) \in im(f)$, $xy = f(a)f(b) = f(ab) \in im(f)$ 
        \item for any $x=f(a) \in im(f)$, $x^{-1} = f(a)^{-1} = f(a^{-1}) \in im(f)$
    \end{itemize}
\end{proof}

\begin{boxedProblemExtra}
    Let $V$ denote the Klein 4-group. Show that $Aut(V)$ is isomorphic to $S_3$.
\end{boxedProblemExtra}

\begin{boxedProblemExtra}
    Define f: $GL_n(\R)$ is the transpose of $A$. Show that $f$ is an automorphism, but not an inner automorphism for $n \geq 1$. 
\end{boxedProblemExtra}
\noindent\rule{\textwidth}{1pt}